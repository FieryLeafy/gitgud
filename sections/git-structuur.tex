\section[Structuur]{Structuur van Git}

\subsection{Commits}
\begin{frame}{Commits}
	Geschiedenis opgebouwd uit commits: vastgelegde punten in de geschiedenis van je programma.
	\begin{itemize}
		\item Een commit is een snapshot
		\item Elke commit verwijst naar zijn voorganger(s)
		\item Elke commit krijgt een unieke ID van 40 tekens
		\item Zodra een commit is gedeeld (push) niet meer aan te passen
	\end{itemize}

	Bijna alles wat je met Git doet is commits aanmaken.
\end{frame}

\subsection{Commits}
\begin{frame}{Workflow}
	Om geschiedenis te schrijven / commits te maken gebruik je de ``Edit-Add-Commit''-cyclus:
	\begin{enumerate}
		\item Bewerk bestanden (meestal tot `werkend' punt)
		\item \texttt{git add}: Bestanden naar staging area
		\begin{itemize}
			\item \texttt{git status}, \texttt{git diff --staged}
		\end{itemize}
		\item Zeker? $\rightarrow$ \texttt{git commit}
	\end{enumerate}
\end{frame}
