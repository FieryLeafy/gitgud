\section[W\&H?]{Waarom \& hoe Git?}

\subsection{Version control}
\begin{frame}{Wat is version control?}
	Op een niet-onhandige manier:
		\begin{itemize}
			\item Code delen met meerdere mensen
			\item Geschiedenis bijhouden en terugdraaien
			\item Veranderingen (evt. over langere tijd) zichtbaar maken
			\item Overzicht van wie wat veranderde
		\end{itemize}
\end{frame}

\subsection{Waarom Git?}
\begin{frame}{Waarom Git?}
	\begin{itemize}
		\item Distributed:
			\begin{itemize}
				\item Je hebt zelf (standaard) de gehele geschiedenis
				\item Je kan zonder internet werken
			\end{itemize}
		\item Integrity: bestanden kunnen niet veranderen zonder dat Git het merkt
		\item Moeilijk om data te vernietigen (`kwijt' kan, maar dan moet je beter opletten)
		\item Non-lineair: kan veel dingen naast elkaar starten (en weer weggooien)
		\item Veel diensten op gebouwd: Github, Travis, \ldots
	\end{itemize}
\end{frame}

\begin{frame}{Waarom command line?}
	\begin{itemize}
		\item Hetzelfde over alle platforms
		\item GUIs gebruiken zelfde namen voor dingen
		\item GUI kapot: `ja ga maar command line'
		\item Uitgebreide handleiding: \texttt{git help}
	\end{itemize}
	Goede GUIs: zie \href{https://git-scm.com/downloads/guis}{de site van Git} en de cheatsheet
	\begin{itemize}
		\item Github Desktop
		\item SmartGit \url{https://syntevo.com/smartgit/}
		\item Gitkraken \url{https://gitkraken.com}
	\end{itemize}
	Als je de command line snapt, snap je de andere clients!
\end{frame}
