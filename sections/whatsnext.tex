\section[Hoe verder?]{Hoe verder na deze tutorial?}

\begin{frame}{Mogelijkheden}
	Git heeft heel veel mogelijkheden voor verschillende use-cases:
	\begin{center}
		\begin{tabular}{ll}
			\texttt{alias} & Verkort veelgebruikte commando's \\
			\texttt{grep}	& Doorzoek alle bij Git bekende bestanden \\
			\texttt{submodules} & Sluit een ander volledig git-repository in \\
			\texttt{stash} & Sla wijzigingen tijdelijk op\\
			Hooks & Voer scripts uit wanneer je iets doet in Git \\
			Travis CI & Bouw en publiceer iets bij elke \texttt{git push}
		\end{tabular}
	\end{center}
	Lees de handleiding: \texttt{git help [commando]}
\end{frame}

\begin{frame}{Extra hulpbronnen}
	\begin{itemize}
		\item Pro Git, het offici\"ele boek: \url{https://git-scm.com/book}
		\item Interactieve tutorials:
			\begin{itemize}
				\item \url{https://try.github.io}
				\item \url{https://codecademy.com/learn/learn-git}
			\end{itemize}
		\item \url{https://www.atlassian.com/git/tutorials/}
		\item \emph{Google} en zelf doen!
	\end{itemize}
\end{frame}

\begin{frame}{git alias}
	Maak verkortingen voor commando's:
	\begin{itemize}
		\item \texttt{git config alias.overzicht "log --all --graph --decorate --oneline"}
		\item \texttt{git config alias.unstage "reset HEAD"}
	\end{itemize}
	Maak aliases global (voor al je git repo's op die computer) met \texttt{git config --global \ldots}
\end{frame}

\begin{frame}{git grep}
	Zoeken in alle tracked files:
	\begin{itemize}
		\item \texttt{git grep "Zoeken"}
		\item \texttt{git grep -i "zOEkEn"} hoofdletterongevoelig
		\item \texttt{git grep -n "Zoeken"} regelnummer
		\item \texttt{git config --global alias.zoek "grep -ni"} alias voor beide
	\end{itemize}
\end{frame}

\begin{frame}{Dat was het dan}
	De slides zijn op
\end{frame}
