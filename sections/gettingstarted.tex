\section{Basisacties}

\subsection{Repository aanmaken}
\begin{frame}[fragile]{Repository maken}
	\begin{enumerate}
		\item Open terminal (windows: \alert{git-bash})
		\item \texttt{cd} naar map die je wil bijhouden
			(of \texttt{mkdir} een nieuwe)
		\item \texttt{git init}
		\item \texttt{git status}
	\end{enumerate}
	Als het goed is zie je nu \alert{niet}:
	\begin{minted}[bgcolor=light-gray]{text}
fatal: Not a git repository or
	any of the parent directories: .git
	\end{minted}
\end{frame}

\subsection{Bestanden laten bijhouden}
\begin{frame}[fragile]{Je eerste commit}
	\begin{enumerate}
		\item \texttt{git status}: overzicht van wat afwijkt van opgeslagen
		\item Bekijk 'untracked files'
		\item \texttt{git add bestand1 bestand2 map1}\\ of:
			\texttt{git add .}\\
			(opgeven van map pakt alles erin, . is huidige map)
		\item \texttt{git commit}, voer bericht in, opslaan en sluiten\\
			(of: \texttt{git commit -m "Eerste commit"})
	\end{enumerate}
	Resultaat: 
	\begin{minted}[bgcolor=light-gray]{text}
[master 1234abc] Eerste commit
x files added
	\end{minted}
\end{frame}

\begin{frame}{Achter de schermen}
	\begin{enumerate}
		\item Beginsituatie voor de graaf gemaakt, 0 nodes
		\item Bestanden in staging area geplaatst
		\item Nieuwe node aan de graaf toegevoegd met 0 verwijzingen
	\end{enumerate}
	\begin{center}
		\includegraphics[width=.3\textwidth]{graphs/basis0.eps}
	\end{center}
\end{frame}

\begin{frame}{Bestanden ontstagen}
	\texttt{git add} is ongedaan te maken:\\
	\texttt{git reset HEAD <bestand>}\\
	(Zie ook de tekst van \texttt{git status})
\end{frame}

\begin{frame}{Help mijn commit is niet goed}
	\texttt{git commit --amend}:
	\begin{itemize}
		\item Commit message nog aanpassen
		\item Files die niet gestaged waren
	\end{itemize}
	\alert{Commit moet nieuwste en niet gedeeld zijn}
\end{frame}

\subsubsection{Wat is hier gebeurd?}
\begin{frame}{Wat is er nu eigenlijk gebeurd?}
	\begin{center}
		\includegraphics[width=.7\textwidth]{graphs/basis1.eps}<1>
		\includegraphics[width=.7\textwidth]{graphs/basis2.eps}<2>
	\end{center}
\end{frame}

\subsubsection{Commit messages}
\begin{frame}{Commit message}
	Iedere commit heeft een message:
	\begin{itemize}
		\item Subject line
		\item Lege regel
		\item Body
	\end{itemize}
\end{frame}

\begin{frame}{Commit message schrijven}
	Niet verplicht, wel handig/netjes:
	\begin{enumerate}
		\item Scheid subject en body met een lege regel (!)
		\item Subject niet langer dan 50 tekens
		\item Subject beginnen met een hoofdletter
		\item Subject niet eindigen met een punt
		\item Subject imperatief: `Verwijder alles' i.p.v. `Verwijdert alles'
		\item Regels van de body op 72 tekens lang
		\item Zet \emph{wat} en \emph{waarom} in de body, \emph{hoe} gebeurt automatisch
	\end{enumerate}
	% Bron: http://chris.beams.io/posts/git-commit/
\end{frame}

\begin{frame}
	\begin{center}
		\includegraphics[width=\textwidth]{images/areas}
	\end{center}
	\begin{itemize}
		\item \texttt{git add} $\rightarrow$ 'Stage Fixes'
		\item \texttt{git commit} $\rightarrow$ 'Commit'
	\end{itemize}
\end{frame}

\subsubsection{Bestanden negeren}
\begin{frame}[fragile]{Bestanden negeren}
	\begin{itemize}
		\item \texttt{git status} geeft onbekende bestanden altijd aan
		\item Oplossing 1: \texttt{.gitignore}:
	\end{itemize}
	\begin{minted}[bgcolor=light-gray]{text}
# alle bestanden in de map bin
bin/*

# alle .exe
*.exe

# maar wel dinges.exe
!dinges.exe
	\end{minted}

	(\texttt{.gitignore} moet wel gecommit worden)
\end{frame}

\begin{frame}{Bestanden alleen voor jezelf negeren}
	\texttt{.git/info/exclude}:
	\begin{itemize}
		\item Alleen op je eigen kopie van repo
		\item Zelfde syntax als \texttt{.gitignore}
	\end{itemize}

	Wat te negeren:
	\begin{itemize}
		\item Gegenereerde resultaten (binaries, logs)
		\item Gevoelige informatie (wanneer gedeeld)
		\item Heel grote bestanden die veel veranderen (Github LFS)
	\end{itemize}
\end{frame}

\subsection{Geschiedenis bekijken}
% git log
\begin{frame}{log}
	\begin{tabular}{ll}
		\texttt{git log}& Bekijk geschiedenis van commits (\texttt{--reverse})\\
		\texttt{git log id1..id2} & Alle geschiedenis tussen \texttt{id1} en \texttt{id2}\\
		\texttt{git show}& Bekijk veranderingen in commit, standaard nieuwste
	\end{tabular}

	De ID's uit \texttt{log} kan je gebruiken in \texttt{git diff}:
\end{frame}

% git diff
\begin{frame}{git diff}
	\begin{tabular}{ll}
		\texttt{git diff}&Toon veranderingen in tracked files (unstaged)\\
		\texttt{git diff --staged}&Toon veranderingen klaargezet voor commit\\
		\texttt{git diff HEAD\^}&Toon veranderingen sinds vorige commit\\
		\texttt{git diff id1 id2}&Toon verschil tussen commit \texttt{id1} en \texttt{id2}
	\end{tabular}
	{\footnotesize Waar je id's uit \texttt{git log} haalt. }
\end{frame}

% git show
\begin{frame}{git show}
	\begin{tabular}{l l}
		\texttt{git show}&Toon info over vorige commit\\
		\texttt{git show id}&Toon info over commit \texttt{id}\\
	\end{tabular}
\end{frame}

\subsection{Overzicht}
\begin{frame}{Overzicht (doemoment!)}
	{ \footnotesize
	\begin{tabular}{ll}
		\texttt{git init}					& maak een nieuwe repository \\
		\texttt{git status} 				& bekijk wat git denkt dat er aan de hand is \\
		\texttt{git add file map \ldots}	& bestanden klaarzetten voor commit (stagen)	\\
		\texttt{git commit} 				& klaargezette bestanden in geschiedenis zetten (committen)\\
		\texttt{git reset file}		    	& bestanden ontstagen (omgekeerde van \texttt{add})	\\
		\hline
		\texttt{git diff}					& Veranderingen in tracked files					\\
		\texttt{git diff --staged}			& Wat ga ik committen? (wat is staged?)				\\
		\texttt{git diff HEAD\^}			& Wat is er veranderd sinds de vorige commit?		\\
		\hline
		\texttt{git log}					& Bekijk vorige commits								\\
		\texttt{git show id}				& Bekijk commit \texttt{id} in detail
	\end{tabular}
	}
	Doen:
	\begin{enumerate}
		\item Maak een lege map met repo
		\item Download \url{https://j.mp/gitgud-ex1}
		\item Add en commit
		\item Bug: 16 moet \texttt{`date +\%d`} zijn, repareer en commit
		\item \texttt{bash simpelscript.sh}
	\end{enumerate}
\end{frame}

\section[Ongedaan]{Dingen ongedaan maken}

\subsection{Bestand terugdraaien naar onbewerkt}
\begin{frame}{Bestand terug naar vorige commit}
	\texttt{git checkout -- <bestand>} \\
	\alert{Omdat dit niet gecommit was ben je je wijzigingen kwijt!}
\end{frame}

\subsection{Bestanden terugdraaien naar eerdere versie}
\begin{frame}{Bestand(en) terug naar eerdere versie}
	\begin{itemize}
		\item \texttt{git checkout <hash> <bestand>}: enkele bestanden, staget
		\item \texttt{git checkout HEAD <bestand>}: vorige ongedaan maken
		\item \texttt{git checkout <hash>}: detached HEAD
		\item \texttt{git checkout master}: detached HEAD ongedaan maken
	\end{itemize}
	%(Hash opzoeken met git log, of: \texttt{HEAD\^, HEAD\^\^, HEAD\textasciitilde 5})
\end{frame}

\begin{frame}{Detached HEAD in plaatjes}
	\begin{center}
		\includegraphics[width=.7\textwidth]{graphs/basis3.eps}<1,3>
		\includegraphics[width=.7\textwidth]{graphs/basis4.eps}<2>
	\end{center}
	\only<1>{Basisstaat van eerder}
	\only<2>{\texttt{git checkout B}}
	\only<3>{\texttt{git checkout master} (niet \texttt{C}!)}
\end{frame}

\subsection{Commits ongedaan maken}
\begin{frame}{git revert}
	\begin{itemize}
		\item \texttt{git revert <hash>}: maak nieuwe omgekeerde commit
		\item Safe: reverts van reverts van reverts kan je reverten
	\end{itemize}
\end{frame}

\begin{frame}{git reset}
	Veilig:
	\begin{itemize}
		\item \texttt{git reset}: unstage alles
		\item \texttt{git reset <bestand>}: unstage bestand
	\end{itemize}
	\alert{Onveilig:}
	\begin{itemize}
		\item \texttt{git reset --hard}: \texttt{checkout --} op hele working directory
		%\item \texttt{git reset <commit>}: verwijder alle commits na \texttt{<commit>}, laat WD staan
		%\item \texttt{git reset --hard <commit>}: \alert{dat is pech, data weg!}
	\end{itemize}
	%\alert{Reset alleen lokale dingen!}
\end{frame}

\begin{frame}{git clean}
	\alert{Verwijdert untracked bestanden!}
	\begin{itemize}
		\item \texttt{git clean -n}: dry run
		\item \texttt{git clean -x}: inclusief genegeerd
		\item \texttt{git clean -d}: untracked mappen
		\item \texttt{git clean --force}: \alert{poef}
	\end{itemize}
	(Soms beter om een snapshot van je branch te downloaden)
\end{frame}

\subsection{Overzicht}
\begin{frame}{Overzicht}
	{ \footnotesize
	\begin{tabular}{ll}
		\texttt{git checkout -- bestand}		& bestand terug naar vorige commit	\\
		\texttt{git commit} & klaargezette bestanden in geschiedenis zetten (committen)\\
		\texttt{git reset}	& bestanden ontstagen (omgekeerde van \texttt{add}
	\end{tabular}
	}
	Doen:
	\begin{enumerate}
		\item Ga terug naar je repo van eerder
		\item Run het script (\texttt{bash simpelscript.sh}) $\rightarrow$ untracked file
		\item \texttt{git clean -n}, \texttt{git clean -f}
		\item \texttt{bash simpelscript.sh}
		\item \texttt{.gitignore} maken met \texttt{laatstekeer.txt}, \texttt{git add .gitignore}
		\item \texttt{git clean -n} : bestand weg?
		\item \texttt{git clean -xn}, \texttt{git clean -xf}
		\item Commit je \texttt{.gitignore}
	\end{enumerate}
	Cheat sheet van eerdere dingen? \url{http://overapi.com/git}
\end{frame}
