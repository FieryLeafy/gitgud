\section{Git downloaden en instellen}

\subsection{Downloaden en installeren}
Als eerste zal je Git moeten installeren voor je het kan gebruiken. Git is te
downloaden van de offici\"ele site, \url{https://git-scm.com/downloads}.

Hiernaast is het aan te raden om ook een GUI-client te installeren, zodat je
niet eeuwig commando's blijft invoeren. Dit is niet vereist voor deze opdrachten
(er wordt bijna alleen maar command-line gebruikt, omdat dit op alle platforms
werkt), maar wordt zeker door nieuwere gebruikers als prettiger ervaren.

Een compleet overzicht van de clients voor jouw platform kan je bekijken op
\url{https://git-scm.com/downloads/guis}, clients die op alle drie de platforms
(Windows, macOS, Linux) werken en worden aangeraden zijn:

\begin{itemize}
	\item SmartGit, \url{https://www.syntevo.com/smartgit/}
	\item GitKraken, \url{https://www.gitkraken.com/}
\end{itemize}

\subsection{Nuttige instellingen}
Git heeft wat informatie nodig om zijn werk te kunnen doen, en hiernaast zijn er
ook een aantal configuratieopties die het leven makkelijker maken.

De instellingen van Git worden beheerd via het commando \cmd{git config}.
Instellingen kunnen op twee niveaus worden aangepast: voor het repository waar
je in aanwezig bent (dus voor \'e\'en project), of voor alle projecten die je
met jouw gebruiker bewerkt. Wanneer je een instelling hebt aangepast krijg je
standaard geen bevestiging dat het is gelukt (geen tekst en gelijk weer
een nieuw commando kunnen invoeren), dit is normaal.

Om een instelling op te vragen kan je \cmd{git config --get $instelling}
uitvoeren, om de waarde die aan \cmd{$instelling} is gekoppeld op te vragen. Als
er geen waarde is ingesteld zal geen tekst worden weergegeven.

Om een instelling voor het huidige project aan te passen, moet je in een map die
door Git wordt beheerd (\emph{repository}) zitten, en dan het commando \cmd{git
config $instelling $waarde} uitvoeren, waarbij je \cmd{$instelling} en
\cmd{$waarde} vervangt door de naam van de instelling en de waarde die de
instelling moet krijgen.

Als je een instelling voor al je repositories wilt aanpassen, kan je in een
willekeurige map het commando \cmd{git config --global $instelling $waarde}
uitvoeren. Globale instellingen worden vervangen door plaatselijke instellingen,
hierdoor kan je bijvoorbeeld voor het ene repository je universiteitsmailadres
gebruiken en voor een ander project je persoonlijke mailadres.

\subsubsection{Verplichte instellingen}
De volgende instellingen zijn verplicht, en kunnen zoals boven beschreven worden
aangepast met het commando \cmd{git config}. Het is aan te raden om dit met de
\texttt{--global}-optie uit te voeren, omdat je anders voor ieder project dit
moet instellen.

\begin{center}
\begin{tabular}{ll}
	\textbf{Instelling}	& \textbf{Waarde} \\ \hline
	\texttt{user.name} & De naam die je aan je commits wilt verbinden.\\
	\texttt{user.email} & Het emailadres dat je aan je commits wilt verbinden.
\end{tabular}
\end{center}

Deze instellingen zijn verplicht omdat de waarden die je hier instelt aan alle
commits die je maakt worden verbonden! Denk dus goed na over wat je hier
invoert, want als je eenmaal geschiedenis hebt geschreven is het niet mogelijk
om dit gemakkelijk aan te passen! (Het \emph{kan} wel, maar levert heel veel
gezeur op\ldots)

Bij het instellen moeten de waarden ook met aanhalingstekens worden omringd, dus
zo:
\begin{minted}{shell-session}
$ git config --global user.name "Jouw naam hier"
$ git config --global user.email "grappig@email.adres"
\end{minted}

\subsubsection{Handige instellingen}
De volgende instellingen zijn niet verplicht, maar maken het leven wel
makkelijker:

\begin{center}
	\begin{tabular}{p{.2\textwidth}p{.8\textwidth}}
		\textbf{Instelling} & \textbf{Waarde} \\ \hline
		\texttt{core.editor} & Zie \ref{kieseditor}, stelt de standaard
		teksteditor in.\\
		\texttt{color.ui} & \cmd{auto} voor kleuren in de uitvoer van Git
		(leest makkelijker) \\
		\texttt{credential.helper} & Zet eenmaal op \cmd{"cache"}, en dan op
		\cmd{"cache --timeout=3600"} \footnotemark{}
	\end{tabular}
	\footnotetext{Dit zorgt ervoor dat Git een uur lang je wachtwoord onthoudt
		als je dat hebt ingevoerd.}

\end{center}

\subsubsection{\texttt{core.editor} instellen}
\label{kieseditor}
Standaard gebruikt Git een enigzins `aparte' editor om je berichten in te laten
voeren en tekst direct te laten bewerken, namelijk Vim. Dit is ontzettend leuk
als je weet hoe het werkt, maar anders krijg je waarschijnlijk niet veel nuttigs
eruit.

Op macOS en Linux kan je de waarde veilig op \cmd{nano} zetten, dit roept een
standaard-teksteditor aan die wel geschikt is voor nieuwe gebruikers. Als je een
wat geavanceerdere editor wilt hebben kan je Atom installeren
(\url{https://atom.io}), en vervolgens de waarde op \cmd{"atom --wait"}
zetten (inclusief aanhalingstekens).

Op Windows is het iets lastiger om het te laten werken, je hebt namelijk het
volledige pad naar je editor nodig. (Behalve als je Kladblok/Notepad goed genoeg vindt
werken, dan kan je \texttt{notepad} instellen.) Hieronder worden de waarden voor
een aantal veelgebruike editors weergegeven, als je een andere editor hebt of
wilt gebruiken zal je hier zelf naar op zoek moeten gaan.

(Verander deze waarde dus door \cmd{git config --global core.editor $waarde}
uit te voeren.)

\begin{center}
	\begin{tabular}{p{0.2\textwidth}p{0.8\textwidth}}
		\textbf{Editor} & \textbf{Waarde} \\ \hline
		Notepad++ & \cmd{"'C:/Program Files/Notepad++/notepad++.exe'
			-multiInst -notabbar -nosession -noPlugin"} \\
		Notepad++, 64-bits systeem & \cmd{"'C:/Program Files
			(x86)/Notepad++/notepad++.exe' -multiInst -notabbar -nosession
			-noPlugin" }\\
		Atom & \cmd{"atom --wait"}
	\end{tabular}
\end{center}

\subsection{TL;DR}
\begin{enumerate}
	\item Stel je naam en emailadres in via \cmd{git config --global user.name}
		en \cmd{git config --global user.email}.
	\item Zet je editor op iets anders dan Vim: \cmd{git config --global
		core.editor $waarde}
\end{enumerate}
